\documentclass{chi-ext}
% Please be sure that you have the dependencies (i.e., additional LaTeX packages) to compile this example.
% See http://personales.upv.es/luileito/chiext/

\copyrightinfo{
  Copyright is held by the author/owner(s).\\
  \emph{CHI'13}, April 27 -- May 2, 2013, Paris, France.\\
  ACM 978-1-XXXX-XXXX-X/XX/XX.\\
}

\title{Simple Visual and Vibrotactile Patterning within Transmedia Gaming Experiences}

\numberofauthors{2}
% Notice how author names are alternately typesetted to appear ordered in 2-column format;
% i.e., the first 4 authors on the first column and the other 4 authors on the second column.
% Actually, it's up to you to strictly adhere to this author notation.
\author{
  \alignauthor{
  	\textbf{Adam Tindale, Ph.D.}\\
  	\affaddr{OCAD University}\\
  	\affaddr{Toronto, ON M5T 1W1 Canada}\\
  	\email{atindale@faculty.ocadu.ca}
  }\alignauthor{
  	\textbf{Michael Cumming, Ph.D.}\\
  	\affaddr{OCAD University}\\
  	\affaddr{Toronto, ON M5T 1W1 Canada}\\
  	\email{mcumming@ocadu.ca}
  }
}

\teaser{
  \includegraphics[width=\columnwidth]{images/P1130386.jpg}
  \caption{A simple wearable device prototype that features visual and vibrotactile sensory outputs for transmedia users.}
  \label{fig:rubberVibeBand01}
}

% Paper metadata (use plain text, for PDF inclusion and later re-using, if desired)
\def\plaintitle{Simple Visual and Vibrotactile Patterning within Transmedia Gaming Experiences}
\def\plainauthor{Adam Tindale, Ph.D.}
\def\plainkeywords{pattern recognition, wearable devices, gaming, transmedia, multi-sensory, vibrotactile}
\def\plaingeneralterms{Gaming, Patterns, Transmedia}

\hypersetup{
  % Your metadata go here
  pdftitle={\plaintitle},
  pdfauthor={\plainauthor},  
  pdfkeywords={\plainkeywords},
  pdfsubject={\plaingeneralterms},
  % Quick access to color overriding:
  citecolor=black,
  linkcolor=blue,
  menucolor=black,
  urlcolor=blue,
}

\usepackage{graphicx}   % for EPS use the graphics package instead
\usepackage{balance}    % useful for balancing the last columns
\usepackage{bibspacing} % save vertical space in references

\begin{document}

\maketitle

\begin{abstract}
Transmedia games are notable for the possibilities they afford for rich multimedia narratives and innovative ways of integrating and switching between diverse media types. Simple, wearable devices are a promising way of encouraging participation with those who consume and contribute towards transmedia narratives. In this paper we explore techniques for adding simple visual and vibro-tactile patterning for children aged 8-12, which add sensory interest to gameplay as well as convey useful information about basic game dynamics and expected user interactions.
\end{abstract}

\keywords{\plainkeywords}
\textcolor{red}{Mandatory section to be included in your final version.}

\category{H.5.m}{Information interfaces and presentation (e.g., HCI)}{Miscellaneous}. 
%See \cite{ACMCCS} 
See: \url{http://www.acm.org/about/class/1998/} 
\textcolor{red}{Mandatory section to be included in your final version.}

\terms{\plaingeneralterms}
\textcolor{red}{Optional section to be included in your final version.}


\section{Introduction}
This project involves adding simple user involvement and user participation to complex transmedia narratives. Transmedia narratives are by their nature complex and multisensory. If well done, they are engaging to their viewers and can build interesting social cohorts quickly. They have the potential of explaining complex concepts and making connections between concepts in a fluid and graceful way. Yet, such narratives demand a high level of skill in their artistic conception and organization. 

Users wish to interact with these narratives in order to direct the narrative, to provide information demanded by the game and to acknowledge that have sufficient involvement to proceed in the game. Such interactions could easily be confusing and counter-productive to the flow of the narrative. 

Transmedia attempts to switch quickly between diverse media but in a way that is interesting and engaging with their audience. It is the assumption here that adding complex user interaction on top of the sometimes challenging demands of transmedia is likely be unsuccessful and confusing to the audience. Instead, we take a different approach: we try to make the interactions very simple, such that they convey information in a way that presents few cognitive demands on the user and which the user would tend to find enjoyable and engaging even if they were not participating in a transmedia game. 

\section{Transmedia Gaming and Narratives}
Transmedia games involve interacting with complex narratives in a simple way that doesn't confuse its viewers. The complexity of the narrative should match the complexity of the user interaction expected of the viewer. It shouldn't be overly complex, but just right.

\section{Interaction Points within Narratives}
There are certain points within transmedia narratives that benefit from user interaction. These enable the viewer to engage with the game by making a choice, by stating a preference and by directing in the course of the narrative. 


\section{Visual and Vibrotactile Patterning}
Patterns need not be complex to be both engaging and useful conveyers of information. The simplest patterns are often the most compelling. 

\section{How Patterns Can Affect Gameplay}
Patterns can be engaging on their own?they can form games which themselves are capable of transmedia experiences?and they can also function as an integral parts of larger technical, artistic and social gameplay worlds. 

\section{Conclusions}
Transmedia narrative are technical constructions, but they are also depend on subtle artistic  techniques that switch between narratives in a seamless and fluid manner. We believe that the final quality of a transmedia narrative ultimately depends on this kind of artistic judgement.

\section{Future Work}
Wearable devices and associated technologies for developing is developing at a rapid pace.
Miniaturization, bearability and functionality is obviously a primary concern of such devices. 


\section{References and Citations}

\section{Acknowledgements}
We thank the valuable input from Patrick Crowe and all those at Xenophile Media Inc. who have great expertise in crafting transmedia narratives; Dr. Rachel Zuannon from the Graduate Design Program at Anhembi Morumbi University, Sao Paulo, Brazil.; the professors, students and staff within OCAD University's Digital Futures Initiative (DFI) for their valuable input and helpful comments. The following students within the DFI program worked on this project over many months: Hudson Pridham, Ryan Maksymic, Jessica Peter and Boris Kourtoukov. We thank Dr. Steve Szigeti for his contributions to this project and we especially thank Dr. Sara Diamond, President of OCAD University for her guidance and support.

\balance
\bibliographystyle{acm-sigchi}
\bibliography{sample}

\end{document}